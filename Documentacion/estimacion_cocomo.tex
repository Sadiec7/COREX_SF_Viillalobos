\documentclass[12pt]{article}
\usepackage[spanish]{babel}
\usepackage[T1]{fontenc}
\usepackage{geometry}
\usepackage{amsmath}
\usepackage{booktabs}
\usepackage{array}
\geometry{margin=2.5cm}

\title{Estimaci\'on de esfuerzo y costo con COCOMO}
\author{Equipo de desarrollo}
\date{\today}

\begin{document}
\maketitle

\section{Contexto del proyecto}
El sistema de gesti\'on de seguros se desarrolla con Electron siguiendo una arquitectura MVC (Model-View-Controller). 
El alcance actual considera controladores, modelos y vistas para autenticaci\'on, cat\'alogos, clientes, p\'olizas, recibos y documentos. 
El objetivo de este documento es dejar trazabilidad de los supuestos y resultados de la estimaci\'on de esfuerzo y costo aplicando el modelo COCOMO (Constructive Cost Model) b\'asico.

\section{Datos de entrada}
Se realiz\'o un conteo de l\'ineas de c\'odigo (LOC, por sus siglas en ingl\'es) con la utiler\'ia \texttt{wc -l} sobre los directorios de trabajo (\texttt{controllers}, \texttt{models} y \texttt{views}), excluyendo dependencias externas.
Los resultados se resumen en la Tabla~\ref{tab:loc}.

\begin{table}[h]
    \centering
    \begin{tabular}{l r}
        \toprule
        Componente & LOC (l\'ineas de c\'odigo) \\
        \midrule
        Controladores & 3\,480 \\
        Modelos & 2\,374 \\
        Vistas & 2\,877 \\
        \midrule
        Total & 8\,731 \\
        \bottomrule
    \end{tabular}
    \caption{Conteo aproximado de l\'ineas fuente entregables.}
    \label{tab:loc}
\end{table}

El conteo total equivale a $KLOC = 8.731$ (miles de l\'ineas de c\'odigo). 
El c\'odigo HTML (HyperText Markup Language) y CSS (Cascading Style Sheets) de las vistas se consider\'o parte del producto entregable al cliente interno, por lo que se incluye en el c\'alculo.

\subsection{Puntos funcionales}
Adem\'as del conteo en l\'ineas, se catalogaron 15 puntos funcionales (PF, \textit{Functional Points}) a partir de los requerimientos documentados: autenticaci\'on, gesti\'on de clientes, p\'olizas, recibos, cat\'alogos, auditor\'ia y documentos.
La secci\'on~\ref{sec:factores-funcionales} detalla la valoraci\'on de los factores de influencia asociados a estos PF.

\section{Selecci\'on del modo COCOMO}
Se elige el modo \emph{org\'anico} del COCOMO b\'asico por las siguientes razones:
\begin{itemize}
    \item El alcance descrito en \textit{ESPECIFICACIONES\_COMPLETAS.md} se limita a operaciones CRUD (Create, Read, Update, Delete) locales para un corredor de seguros, sin integraciones externas ni requisitos de escalamiento.
    \item El equipo de desarrollo est\'a conformado por cuatro estudiantes sin experiencia profesional previa, lo que alinea con las premisas del modo org\'anico.
    \item El sistema se desplegar\'a en una sola estaci\'on de trabajo y se anticipa una comunicaci\'on directa con el usuario final, caracter\'istica prevista en el modo org\'anico.
\end{itemize}

Los coeficientes empleados son los propuestos por Boehm para dicho modo: $a = 2.4$, $b = 1.05$, $c = 2.5$ y $d = 0.38$.

\section{Resultados}
El esfuerzo estimado en meses-persona (PM) es
\[
E = a \times (KLOC)^{b} = 2.4 \times (8.731)^{1.05} = 23.35 \text{ PM}.
\]

El tiempo de desarrollo en meses resulta de
\[
T = c \times (E)^{d} = 2.5 \times (23.35)^{0.38} = 8.28 \text{ meses}.
\]

El tama\~no promedio del equipo requerido es
\[
P = \frac{E}{T} = \frac{23.35}{8.28} = 2.82 \text{ personas}.
\]

Para valorar el costo se asume un estipendio de \$20\,000 MXN (pesos mexicanos) por mes-persona acorde a un equipo estudiantil que opera en un entorno local, sin costos corporativos adicionales. 
Con este supuesto el costo directo estimado es
\[
\begin{aligned}
C &= E \times 20\,000 = \$467\,045 \text{ MXN} \\
  &\approx \$27\,473 \text{ USD (d\'olares estadounidenses) (@ 1 USD = 17 MXN)}.
\end{aligned}
\]

La Tabla~\ref{tab:resumen} resume los indicadores principales.
Adicionalmente, al disponer de cuatro desarrolladores a tiempo completo, el calendario podr\'ia acortarse a $E / 4 = 5.84$ meses, mientras que con una persona dedicada el esfuerzo implicar\'ia aproximadamente $23.35$ meses de trabajo.

\begin{table}[h]
    \centering
    \begin{tabular}{l r}
        \toprule
        Indicador & Valor \\
        \midrule
        Esfuerzo & 23.35 PM \\
        Duraci\'on & 8.28 meses \\
        Tama\~no del equipo & 2.82 personas \\
        Costo estimado & \$467\,045 MXN \\
        \bottomrule
    \end{tabular}
    \caption{Resumen de la estimaci\'on COCOMO en modo org\'anico.}
    \label{tab:resumen}
\end{table}

\section{Escenarios alternos}
Se analiz\'o la sensibilidad del modelo ante variaciones de $\pm 10\%$ en el KLOC, manteniendo los mismos coeficientes. 
Los resultados se muestran en la Tabla~\ref{tab:escenarios}.

\begin{table}[h]
    \centering
    \begin{tabular}{l r r r r}
        \toprule
        Escenario & KLOC & Esfuerzo (PM) & Duraci\'on (meses) & Costo (MXN) \\
        \midrule
        Conservador (-10\%) & 7.86 & 20.91 & 7.94 & \$418\,132 \\
        Base & 8.73 & 23.35 & 8.28 & \$467\,045 \\
        Ambicioso (+10\%) & 9.60 & 25.81 & 8.60 & \$516\,203 \\
        \bottomrule
    \end{tabular}
    \caption{Sensibilidad del esfuerzo y costo ante variaciones en el tama\~no.}
    \label{tab:escenarios}
\end{table}

\section{Factores de ajuste por puntos funcionales}
\label{sec:factores-funcionales}
El conteo de 15 PF se ajust\'o aplicando los 15 factores de influencia sugeridos por IFPUG (International Function Point Users Group), calificando cada uno en la escala de 0 (no aplica) a 5 (influencia fuerte). 
La Tabla~\ref{tab:factores} resume la calificaci\'on y el criterio usado para este proyecto.

\begin{table}[h]
    \centering
    \begin{tabular}{>{\raggedright\arraybackslash}p{5cm} c >{\raggedright\arraybackslash}p{7cm}}
        \toprule
        Factor & Calificaci\'on & Justificaci\'on \\
        \midrule
        Comunicaci\'on de datos & 1 & El sistema opera offline y s\'olo comparte datos mediante reportes locales. \\
        Procesamiento distribuido & 0 & No hay servidores remotos ni componentes distribuidos. \\
        Rendimiento & 1 & El volumen de usuarios y datos es bajo; se requiere respuesta adecuada en una sola estaci\'on. \\
        Configuraci\'on de hardware & 1 & Se ejecuta en equipo tipo oficina, con requisitos modestos descritos en la documentaci\'on. \\
        Volumen de transacciones & 1 & El flujo diario de altas y consultas es reducido y manual. \\
        Entrada de datos en l\'inea & 2 & Formularios interactivos para clientes, p\'olizas y recibos. \\
        Eficiencia para el usuario final & 2 & Se dise\~na una interfaz amigable, pero para una sola persona operadora. \\
        Actualizaci\'on en l\'inea & 1 & Las actualizaciones se aplican localmente sin sincronizaci\'on externa. \\
        Complejidad de procesamiento & 1 & Predominan operaciones CRUD con validaciones b\'asicas y c\'alculos sencillos. \\
        Reusabilidad & 1 & Componentes reutilizables dentro del MVC, pero sin plan de reaprovechamiento en otros proyectos. \\
        Facilidad de instalaci\'on & 3 & Se prioriza un instalador sencillo y aut\'onomo para el usuario final. \\
        Facilidad de operaci\'on & 2 & Se requieren ayudas contextualizadas, pero el flujo operativo es lineal. \\
        M\'ultiples sitios & 0 & El despliegue es exclusivo para la oficina del corredor. \\
        Facilidad de cambios & 2 & Arquitectura modular que facilita ajustes futuros para el negocio. \\
        Seguridad y control de acceso & 2 & Autenticaci\'on local con control de sesiones y resguardo de datos personales. \\
        \bottomrule
    \end{tabular}
    \caption{Calificaci\'on de los 15 factores de influencia para el ajuste de PF.}
    \label{tab:factores}
\end{table}

La suma de las calificaciones es $20$, por lo que el factor de valor de ajuste (VAF) se calcula como
\[
\mathrm{VAF} = 0.65 + 0.01 \times 20 = 0.85.
\]
Tomando los puntos funcionales no ajustados (Unadjusted Function Points, UFP) $= 15$, el puntaje funcional ajustado (Adjusted Function Points, AFP) resulta
\[
\mathrm{AFP} = UFP \times \mathrm{VAF} = 15 \times 0.85 = 12.75.
\]
Este valor se utilizar\'a como referencia para futuros refinamientos del tamaño funcional y su relaci\'on con los estimados obtenidos por COCOMO.

\section{Recomendaciones}
\begin{itemize}
    \item Validar el conteo de l\'ineas con una herramienta especializada (p.~ej. \texttt{cloc}) para refinar la base KLOC.
    \item Ajustar el costo mensual si se conocen salarios reales del equipo o si se desea incluir gastos corporativos adicionales.
    \item Revisar factores de costo (COCOMO intermedio) cuando se cuente con informaci\'on sobre fiabilidad requerida, restricciones de hardware o experiencia del equipo.
\end{itemize}

\end{document}
