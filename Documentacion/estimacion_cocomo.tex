\documentclass[12pt]{article}
\usepackage[spanish]{babel}
\usepackage[T1]{fontenc}
\usepackage{geometry}
\usepackage{amsmath}
\usepackage{booktabs}
\usepackage{array}
\geometry{margin=2.5cm}

\title{Estimaci\'on de esfuerzo y costo con COCOMO}
\author{Equipo de desarrollo}
\date{\today}

\begin{document}
\maketitle

\section{Contexto del proyecto}
El sistema de gesti\'on de seguros se desarrolla con Electron siguiendo una arquitectura MVC (Model-View-Controller). 
El alcance actual considera controladores, modelos y vistas para autenticaci\'on, cat\'alogos, clientes, p\'olizas, recibos y documentos. 
El objetivo de este documento es dejar trazabilidad de los supuestos y resultados de la estimaci\'on de esfuerzo y costo aplicando el modelo COCOMO (Constructive Cost Model) b\'asico.

\section{Datos de entrada}
Se realiz\'o un conteo de l\'ineas de c\'odigo (LOC, por sus siglas en ingl\'es) con la utiler\'ia \texttt{wc -l} sobre los directorios de trabajo (\texttt{controllers}, \texttt{models} y \texttt{views}), excluyendo dependencias externas.
Los resultados se resumen en la Tabla~\ref{tab:loc}.

\begin{table}[h]
    \centering
    \begin{tabular}{l r}
        \toprule
        Componente & LOC (l\'ineas de c\'odigo) \\
        \midrule
        Controladores & 3\,480 \\
        Modelos & 2\,374 \\
        Vistas & 2\,877 \\
        \midrule
        Total & 8\,731 \\
        \bottomrule
    \end{tabular}
    \caption{Conteo aproximado de l\'ineas fuente entregables.}
    \label{tab:loc}
\end{table}

El conteo total equivale a $KLOC = 8.731$ (miles de l\'ineas de c\'odigo). 
El c\'odigo HTML (HyperText Markup Language) y CSS (Cascading Style Sheets) de las vistas se consider\'o parte del producto entregable al cliente interno, por lo que se incluye en el c\'alculo.

\subsection{Puntos funcionales}
Adem\'as del conteo en l\'ineas, se catalogaron 15 puntos funcionales (PF, \textit{Functional Points}) a partir de los requerimientos documentados: autenticaci\'on, gesti\'on de clientes, p\'olizas, recibos, cat\'alogos, auditor\'ia y documentos.
La secci\'on~\ref{sec:factores-funcionales} detalla la valoraci\'on de los factores de influencia asociados a estos PF.

\section{Selecci\'on del modo COCOMO}
Se elige el modo \emph{org\'anico} del COCOMO b\'asico por las siguientes razones:
\begin{itemize}
    \item El alcance descrito en \textit{ESPECIFICACIONES\_COMPLETAS.md} se limita a operaciones CRUD (Create, Read, Update, Delete) locales para un corredor de seguros, sin integraciones externas ni requisitos de escalamiento.
    \item El proyecto presenta caracter\'isticas de baja complejidad con requerimientos estables y bien definidos. El ambiente de desarrollo es familiar y no se requiere innovaci\'on significativa.
    \item El sistema se desplegar\'a en una sola estaci\'on de trabajo y se anticipa una comunicaci\'on directa con el usuario final, caracter\'istica prevista en el modo org\'anico.
    \item El tama\~no del proyecto (8.731 KLOC) se encuentra dentro del rango t\'ipico para proyectos org\'anicos (2-50 KLOC seg\'un la literatura).
\end{itemize}

Los coeficientes empleados son los propuestos por Boehm para dicho modo: $a = 2.4$, $b = 1.05$, $c = 2.5$ y $d = 0.38$.

\section{Resultados}
\subsection{Escenario \'optimo seg\'un COCOMO}
El modelo COCOMO calcula primero el escenario \'optimo de desarrollo, que minimiza costos y maximiza eficiencia.

El esfuerzo total estimado en meses-persona (PM) es
\[
E = a \times (KLOC)^{b} = 2.4 \times (8.731)^{1.05} = 23.35 \text{ PM}.
\]

El tiempo de desarrollo \'optimo en meses resulta de
\[
T = c \times (E)^{d} = 2.5 \times (23.35)^{0.38} = 8.28 \text{ meses}.
\]

El tama\~no \'optimo del equipo requerido es
\[
P = \frac{E}{T} = \frac{23.35}{8.28} = 2.82 \text{ personas}.
\]

\textbf{Interpretaci\'on:} COCOMO recomienda un equipo de \textbf{2.82 personas} trabajando durante \textbf{8.28 meses} como la configuraci\'on \'optima para este proyecto.

\subsection{Ajuste seg\'un recursos disponibles}
El equipo real disponible est\'a conformado por \textbf{4 desarrolladores} a tiempo completo. Con esta configuraci\'on, el tiempo de desarrollo ajustado ser\'ia:
\[
T_{\text{ajustado}} = \frac{E}{P_{\text{disponible}}} = \frac{23.35}{4} = 5.84 \text{ meses}.
\]

Esto representa una aceleraci\'on del 29\% respecto al tiempo \'optimo (de 8.28 a 5.84 meses), lo cual es razonable y no genera sobrecarga excesiva de comunicaci\'on o coordinaci\'on.

Para referencia, con una sola persona dedicada, el desarrollo implicar\'ia aproximadamente $23.35$ meses de trabajo.

\subsection{Costo estimado}
Para valorar el costo se asume un estipendio de \$20\,000 MXN (pesos mexicanos) por mes-persona acorde a un equipo estudiantil que opera en un entorno local, sin costos corporativos adicionales.
Con este supuesto el costo directo estimado es
\[
\begin{aligned}
C &= E \times 20\,000 = \$467\,045 \text{ MXN} \\
  &\approx \$27\,473 \text{ USD (d\'olares estadounidenses) (@ 1 USD = 17 MXN)}.
\end{aligned}
\]

\textbf{Nota:} El costo total se mantiene constante (23.35 personas-mes) independientemente de si se usan 2.82 o 4 personas, ya que representa el esfuerzo total necesario.

\begin{table}[h]
    \centering
    \begin{tabular}{l r r}
        \toprule
        Indicador & Escenario \'optimo & Escenario real (4 personas) \\
        \midrule
        Esfuerzo total & 23.35 PM & 23.35 PM \\
        Duraci\'on & 8.28 meses & 5.84 meses \\
        Tama\~no del equipo & 2.82 personas & 4 personas \\
        Costo estimado & \$467\,045 MXN & \$467\,045 MXN \\
        \bottomrule
    \end{tabular}
    \caption{Comparaci\'on entre el escenario \'optimo calculado por COCOMO y el escenario ajustado con recursos disponibles.}
    \label{tab:resumen}
\end{table}

\section{Limitaciones de escalabilidad: Ley de Brooks}
Es importante comprender que \textbf{no se pueden agregar personas indefinidamente} para acelerar el desarrollo. Frederick Brooks estableci\'o en su obra \textit{The Mythical Man-Month} que:

\begin{quote}
\textit{``Agregar m\'as personas a un proyecto de software retrasado, lo retrasa a\'un m\'as.''}
\end{quote}

\subsection{Factores limitantes}
Existen tres factores principales que limitan la escalabilidad del equipo:

\begin{enumerate}
    \item \textbf{Sobrecarga de comunicaci\'on:} El n\'umero de canales de comunicaci\'on crece como $n(n-1)/2$, donde $n$ es el n\'umero de personas. Con 4 personas hay 6 canales; con 8 personas, 28 canales.

    \item \textbf{Tareas secuenciales:} No todas las tareas pueden paralelizarse. Algunas actividades (an\'alisis de requerimientos, dise\~no de arquitectura, integraci\'on) son inherentemente secuenciales.

    \item \textbf{Tiempo de \textit{ramping}:} Nuevos miembros requieren tiempo para familiarizarse con el c\'odigo, procesos y dominio del problema, reduciendo temporalmente la productividad del equipo.
\end{enumerate}

\subsection{An\'alisis para el proyecto actual}
En el caso de este proyecto:
\begin{itemize}
    \item \textbf{Configuraci\'on:} 4 personas vs 2.82 recomendadas = 42\% m\'as recursos
    \item \textbf{Evaluaci\'on:} Este exceso es \textbf{razonable y manejable} para un proyecto de tama\~no peque\~no-mediano
    \item \textbf{Riesgos:} La sobrecarga de comunicaci\'on es m\'inima con solo 4 desarrolladores (6 canales de comunicaci\'on)
    \item \textbf{Recomendaci\'on:} Mantener comunicaci\'on estrecha mediante reuniones diarias cortas y documentaci\'on clara de interfaces entre componentes
\end{itemize}

Si se intentara acelerar a\'un m\'as agregando 8 o m\'as personas, la eficiencia podr\'ia verse comprometida por la sobrecarga de coordinaci\'on, especialmente en un proyecto con alcance relativamente acotado.

\section{Escenarios alternos}

\subsection{Variaciones en el tama\~no del c\'odigo}
Se analiz\'o la sensibilidad del modelo ante variaciones de $\pm 10\%$ en el KLOC, manteniendo los mismos coeficientes.
Los resultados se muestran en la Tabla~\ref{tab:escenarios}.

\begin{table}[h]
    \centering
    \begin{tabular}{l r r r r}
        \toprule
        Escenario & KLOC & Esfuerzo (PM) & Duraci\'on \'optima (meses) & Costo (MXN) \\
        \midrule
        Conservador (-10\%) & 7.86 & 20.91 & 7.94 & \$418\,132 \\
        Base & 8.73 & 23.35 & 8.28 & \$467\,045 \\
        Ambicioso (+10\%) & 9.60 & 25.81 & 8.60 & \$516\,203 \\
        \bottomrule
    \end{tabular}
    \caption{Sensibilidad del esfuerzo y costo ante variaciones en el tama\~no del c\'odigo.}
    \label{tab:escenarios}
\end{table}

\subsection{Variaciones en el tama\~no del equipo}
Para el escenario base (8.73 KLOC, 23.35 PM de esfuerzo), se analizan diferentes configuraciones de equipo y su impacto en la duraci\'on del proyecto.
La Tabla~\ref{tab:escenarios-equipo} muestra estos escenarios.

\begin{table}[h]
    \centering
    \begin{tabular}{l r r l}
        \toprule
        Configuraci\'on & Personas & Duraci\'on (meses) & Observaciones \\
        \midrule
        1 persona & 1 & 23.35 & Proyecto individual, muy extenso \\
        \'Optimo COCOMO & 2.82 & 8.28 & Configuraci\'on \'optima recomendada \\
        Equipo actual & 4 & 5.84 & Razonable, 42\% m\'as recursos \\
        Equipo ampliado & 6 & 3.89 & Riesgo moderado de sobrecarga \\
        Equipo grande & 8 & 2.92 & Alto riesgo de ineficiencia \\
        \bottomrule
    \end{tabular}
    \caption{Impacto del tama\~no del equipo en la duraci\'on del proyecto (esfuerzo constante: 23.35 PM).}
    \label{tab:escenarios-equipo}
\end{table}

\textbf{Interpretaci\'on:} Aunque matem\'aticamente es posible calcular duraciones menores con equipos m\'as grandes, en la pr\'actica la eficiencia disminuye significativamente con equipos superiores a 4-6 personas para proyectos de este tama\~no, debido a la Ley de Brooks explicada en la secci\'on anterior.

\section{Factores de ajuste por puntos funcionales}
\label{sec:factores-funcionales}
El conteo de 15 PF se ajust\'o aplicando los 15 factores de influencia sugeridos por IFPUG (International Function Point Users Group), calificando cada uno en la escala de 0 (no aplica) a 5 (influencia fuerte). 
La Tabla~\ref{tab:factores} resume la calificaci\'on y el criterio usado para este proyecto.

\begin{table}[h]
    \centering
    \begin{tabular}{>{\raggedright\arraybackslash}p{5cm} c >{\raggedright\arraybackslash}p{7cm}}
        \toprule
        Factor & Calificaci\'on & Justificaci\'on \\
        \midrule
        Comunicaci\'on de datos & 1 & El sistema opera offline y s\'olo comparte datos mediante reportes locales. \\
        Procesamiento distribuido & 0 & No hay servidores remotos ni componentes distribuidos. \\
        Rendimiento & 1 & El volumen de usuarios y datos es bajo; se requiere respuesta adecuada en una sola estaci\'on. \\
        Configuraci\'on de hardware & 1 & Se ejecuta en equipo tipo oficina, con requisitos modestos descritos en la documentaci\'on. \\
        Volumen de transacciones & 1 & El flujo diario de altas y consultas es reducido y manual. \\
        Entrada de datos en l\'inea & 2 & Formularios interactivos para clientes, p\'olizas y recibos. \\
        Eficiencia para el usuario final & 2 & Se dise\~na una interfaz amigable, pero para una sola persona operadora. \\
        Actualizaci\'on en l\'inea & 1 & Las actualizaciones se aplican localmente sin sincronizaci\'on externa. \\
        Complejidad de procesamiento & 1 & Predominan operaciones CRUD con validaciones b\'asicas y c\'alculos sencillos. \\
        Reusabilidad & 1 & Componentes reutilizables dentro del MVC, pero sin plan de reaprovechamiento en otros proyectos. \\
        Facilidad de instalaci\'on & 3 & Se prioriza un instalador sencillo y aut\'onomo para el usuario final. \\
        Facilidad de operaci\'on & 2 & Se requieren ayudas contextualizadas, pero el flujo operativo es lineal. \\
        M\'ultiples sitios & 0 & El despliegue es exclusivo para la oficina del corredor. \\
        Facilidad de cambios & 2 & Arquitectura modular que facilita ajustes futuros para el negocio. \\
        Seguridad y control de acceso & 2 & Autenticaci\'on local con control de sesiones y resguardo de datos personales. \\
        \bottomrule
    \end{tabular}
    \caption{Calificaci\'on de los 15 factores de influencia para el ajuste de PF.}
    \label{tab:factores}
\end{table}

La suma de las calificaciones es $20$, por lo que el factor de valor de ajuste (VAF) se calcula como
\[
\mathrm{VAF} = 0.65 + 0.01 \times 20 = 0.85.
\]
Tomando los puntos funcionales no ajustados (Unadjusted Function Points, UFP) $= 15$, el puntaje funcional ajustado (Adjusted Function Points, AFP) resulta
\[
\mathrm{AFP} = UFP \times \mathrm{VAF} = 15 \times 0.85 = 12.75.
\]
Este valor se utilizar\'a como referencia para futuros refinamientos del tamaño funcional y su relaci\'on con los estimados obtenidos por COCOMO.

\section{Recomendaciones}
\begin{itemize}
    \item Validar el conteo de l\'ineas con una herramienta especializada (p.~ej. \texttt{cloc}) para refinar la base KLOC.
    \item Ajustar el costo mensual si se conocen salarios reales del equipo o si se desea incluir gastos corporativos adicionales.
    \item Revisar factores de costo (COCOMO intermedio) cuando se cuente con informaci\'on sobre fiabilidad requerida, restricciones de hardware o experiencia del equipo.
\end{itemize}

\end{document}
